%%
% Copyright (c) 2017 - 2021, Pascal Wagler;
% Copyright (c) 2014 - 2021, John MacFarlane
%
% All rights reserved.
%
% Redistribution and use in source and binary forms, with or without
% modification, are permitted provided that the following conditions
% are met:
%
% - Redistributions of source code must retain the above copyright
% notice, this list of conditions and the following disclaimer.
%
% - Redistributions in binary form must reproduce the above copyright
% notice, this list of conditions and the following disclaimer in the
% documentation and/or other materials provided with the distribution.
%
% - Neither the name of John MacFarlane nor the names of other
% contributors may be used to endorse or promote products derived
% from this software without specific prior written permission.
%
% THIS SOFTWARE IS PROVIDED BY THE COPYRIGHT HOLDERS AND CONTRIBUTORS
% "AS IS" AND ANY EXPRESS OR IMPLIED WARRANTIES, INCLUDING, BUT NOT
% LIMITED TO, THE IMPLIED WARRANTIES OF MERCHANTABILITY AND FITNESS
% FOR A PARTICULAR PURPOSE ARE DISCLAIMED. IN NO EVENT SHALL THE
% COPYRIGHT OWNER OR CONTRIBUTORS BE LIABLE FOR ANY DIRECT, INDIRECT,
% INCIDENTAL, SPECIAL, EXEMPLARY, OR CONSEQUENTIAL DAMAGES (INCLUDING,
% BUT NOT LIMITED TO, PROCUREMENT OF SUBSTITUTE GOODS OR SERVICES;
% LOSS OF USE, DATA, OR PROFITS; OR BUSINESS INTERRUPTION) HOWEVER
% CAUSED AND ON ANY THEORY OF LIABILITY, WHETHER IN CONTRACT, STRICT
% LIABILITY, OR TORT (INCLUDING NEGLIGENCE OR OTHERWISE) ARISING IN
% ANY WAY OUT OF THE USE OF THIS SOFTWARE, EVEN IF ADVISED OF THE
% POSSIBILITY OF SUCH DAMAGE.
%%

%%
% This is the Eisvogel pandoc LaTeX template.
%
% For usage information and examples visit the official GitHub page:
% https://github.com/Wandmalfarbe/pandoc-latex-template
%%

% Options for packages loaded elsewhere
\PassOptionsToPackage{unicode}{hyperref}
\PassOptionsToPackage{hyphens}{url}
\PassOptionsToPackage{dvipsnames,svgnames*,x11names*,table}{xcolor}
%
\documentclass[
  paper=a4,
  ,captions=tableheading
]{scrartcl}
\usepackage{amsmath,amssymb}
\usepackage{lmodern}
\usepackage{setspace}
\setstretch{1.2}
\usepackage{ifxetex,ifluatex}
\ifnum 0\ifxetex 1\fi\ifluatex 1\fi=0 % if pdftex
  \usepackage[T1]{fontenc}
  \usepackage[utf8]{inputenc}
  \usepackage{textcomp} % provide euro and other symbols
\else % if luatex or xetex
  \usepackage{unicode-math}
  \defaultfontfeatures{Scale=MatchLowercase}
  \defaultfontfeatures[\rmfamily]{Ligatures=TeX,Scale=1}
\fi
% Use upquote if available, for straight quotes in verbatim environments
\IfFileExists{upquote.sty}{\usepackage{upquote}}{}
\IfFileExists{microtype.sty}{% use microtype if available
  \usepackage[]{microtype}
  \UseMicrotypeSet[protrusion]{basicmath} % disable protrusion for tt fonts
}{}
\makeatletter
\@ifundefined{KOMAClassName}{% if non-KOMA class
  \IfFileExists{parskip.sty}{%
    \usepackage{parskip}
  }{% else
    \setlength{\parindent}{0pt}
    \setlength{\parskip}{6pt plus 2pt minus 1pt}}
}{% if KOMA class
  \KOMAoptions{parskip=half}}
\makeatother
\usepackage{xcolor}
\definecolor{default-linkcolor}{HTML}{A50000}
\definecolor{default-filecolor}{HTML}{A50000}
\definecolor{default-citecolor}{HTML}{4077C0}
\definecolor{default-urlcolor}{HTML}{4077C0}
\IfFileExists{xurl.sty}{\usepackage{xurl}}{} % add URL line breaks if available
\IfFileExists{bookmark.sty}{\usepackage{bookmark}}{\usepackage{hyperref}}
\hypersetup{
  hidelinks,
  breaklinks=true,
  pdfcreator={LaTeX via pandoc with the Eisvogel template}}
\urlstyle{same} % disable monospaced font for URLs
\usepackage[margin=2.5cm,includehead=true,includefoot=true,centering,]{geometry}
\usepackage{color}
\usepackage{fancyvrb}
\newcommand{\VerbBar}{|}
\newcommand{\VERB}{\Verb[commandchars=\\\{\}]}
\DefineVerbatimEnvironment{Highlighting}{Verbatim}{commandchars=\\\{\}}
% Add ',fontsize=\small' for more characters per line
\newenvironment{Shaded}{}{}
\newcommand{\AlertTok}[1]{\textcolor[rgb]{1.00,0.00,0.00}{\textbf{#1}}}
\newcommand{\AnnotationTok}[1]{\textcolor[rgb]{0.38,0.63,0.69}{\textbf{\textit{#1}}}}
\newcommand{\AttributeTok}[1]{\textcolor[rgb]{0.49,0.56,0.16}{#1}}
\newcommand{\BaseNTok}[1]{\textcolor[rgb]{0.25,0.63,0.44}{#1}}
\newcommand{\BuiltInTok}[1]{\textcolor[rgb]{0.00,0.50,0.00}{#1}}
\newcommand{\CharTok}[1]{\textcolor[rgb]{0.25,0.44,0.63}{#1}}
\newcommand{\CommentTok}[1]{\textcolor[rgb]{0.38,0.63,0.69}{\textit{#1}}}
\newcommand{\CommentVarTok}[1]{\textcolor[rgb]{0.38,0.63,0.69}{\textbf{\textit{#1}}}}
\newcommand{\ConstantTok}[1]{\textcolor[rgb]{0.53,0.00,0.00}{#1}}
\newcommand{\ControlFlowTok}[1]{\textcolor[rgb]{0.00,0.44,0.13}{\textbf{#1}}}
\newcommand{\DataTypeTok}[1]{\textcolor[rgb]{0.56,0.13,0.00}{#1}}
\newcommand{\DecValTok}[1]{\textcolor[rgb]{0.25,0.63,0.44}{#1}}
\newcommand{\DocumentationTok}[1]{\textcolor[rgb]{0.73,0.13,0.13}{\textit{#1}}}
\newcommand{\ErrorTok}[1]{\textcolor[rgb]{1.00,0.00,0.00}{\textbf{#1}}}
\newcommand{\ExtensionTok}[1]{#1}
\newcommand{\FloatTok}[1]{\textcolor[rgb]{0.25,0.63,0.44}{#1}}
\newcommand{\FunctionTok}[1]{\textcolor[rgb]{0.02,0.16,0.49}{#1}}
\newcommand{\ImportTok}[1]{\textcolor[rgb]{0.00,0.50,0.00}{\textbf{#1}}}
\newcommand{\InformationTok}[1]{\textcolor[rgb]{0.38,0.63,0.69}{\textbf{\textit{#1}}}}
\newcommand{\KeywordTok}[1]{\textcolor[rgb]{0.00,0.44,0.13}{\textbf{#1}}}
\newcommand{\NormalTok}[1]{#1}
\newcommand{\OperatorTok}[1]{\textcolor[rgb]{0.40,0.40,0.40}{#1}}
\newcommand{\OtherTok}[1]{\textcolor[rgb]{0.00,0.44,0.13}{#1}}
\newcommand{\PreprocessorTok}[1]{\textcolor[rgb]{0.74,0.48,0.00}{#1}}
\newcommand{\RegionMarkerTok}[1]{#1}
\newcommand{\SpecialCharTok}[1]{\textcolor[rgb]{0.25,0.44,0.63}{#1}}
\newcommand{\SpecialStringTok}[1]{\textcolor[rgb]{0.73,0.40,0.53}{#1}}
\newcommand{\StringTok}[1]{\textcolor[rgb]{0.25,0.44,0.63}{#1}}
\newcommand{\VariableTok}[1]{\textcolor[rgb]{0.10,0.09,0.49}{#1}}
\newcommand{\VerbatimStringTok}[1]{\textcolor[rgb]{0.25,0.44,0.63}{#1}}
\newcommand{\WarningTok}[1]{\textcolor[rgb]{0.38,0.63,0.69}{\textbf{\textit{#1}}}}

% Workaround/bugfix from jannick0.
% See https://github.com/jgm/pandoc/issues/4302#issuecomment-360669013)
% or https://github.com/Wandmalfarbe/pandoc-latex-template/issues/2
%
% Redefine the verbatim environment 'Highlighting' to break long lines (with
% the help of fvextra). Redefinition is necessary because it is unlikely that
% pandoc includes fvextra in the default template.
\usepackage{fvextra}
\DefineVerbatimEnvironment{Highlighting}{Verbatim}{breaklines,fontsize=\small,commandchars=\\\{\}}

% add backlinks to footnote references, cf. https://tex.stackexchange.com/questions/302266/make-footnote-clickable-both-ways
\usepackage{footnotebackref}
\setlength{\emergencystretch}{3em} % prevent overfull lines
\providecommand{\tightlist}{%
  \setlength{\itemsep}{0pt}\setlength{\parskip}{0pt}}
\setcounter{secnumdepth}{-\maxdimen} % remove section numbering

% Make use of float-package and set default placement for figures to H.
% The option H means 'PUT IT HERE' (as  opposed to the standard h option which means 'You may put it here if you like').
\usepackage{float}
\floatplacement{figure}{H}

\ifluatex
  \usepackage{selnolig}  % disable illegal ligatures
\fi

\author{}
\date{}



%%
%% added
%%

%
% language specification
%
% If no language is specified, use English as the default main document language.
%

\ifnum 0\ifxetex 1\fi\ifluatex 1\fi=0 % if pdftex
  \usepackage[shorthands=off,main=english]{babel}
\else
    % Workaround for bug in Polyglossia that breaks `\familydefault` when `\setmainlanguage` is used.
  % See https://github.com/Wandmalfarbe/pandoc-latex-template/issues/8
  % See https://github.com/reutenauer/polyglossia/issues/186
  % See https://github.com/reutenauer/polyglossia/issues/127
  \renewcommand*\familydefault{\sfdefault}
    % load polyglossia as late as possible as it *could* call bidi if RTL lang (e.g. Hebrew or Arabic)
  \usepackage{polyglossia}
  \setmainlanguage[]{english}
\fi



%
% for the background color of the title page
%

%
% break urls
%
\PassOptionsToPackage{hyphens}{url}

%
% When using babel or polyglossia with biblatex, loading csquotes is recommended
% to ensure that quoted texts are typeset according to the rules of your main language.
%
\usepackage{csquotes}

%
% captions
%
\definecolor{caption-color}{HTML}{777777}
\usepackage[font={stretch=1.2}, textfont={color=caption-color}, position=top, skip=4mm, labelfont=bf, singlelinecheck=false, justification=raggedright]{caption}
\setcapindent{0em}

%
% blockquote
%
\definecolor{blockquote-border}{RGB}{221,221,221}
\definecolor{blockquote-text}{RGB}{119,119,119}
\usepackage{mdframed}
\newmdenv[rightline=false,bottomline=false,topline=false,linewidth=3pt,linecolor=blockquote-border,skipabove=\parskip]{customblockquote}
\renewenvironment{quote}{\begin{customblockquote}\list{}{\rightmargin=0em\leftmargin=0em}%
\item\relax\color{blockquote-text}\ignorespaces}{\unskip\unskip\endlist\end{customblockquote}}

%
% Source Sans Pro as the de­fault font fam­ily
% Source Code Pro for monospace text
%
% 'default' option sets the default
% font family to Source Sans Pro, not \sfdefault.
%
\ifnum 0\ifxetex 1\fi\ifluatex 1\fi=0 % if pdftex
    \usepackage[default]{sourcesanspro}
  \usepackage{sourcecodepro}
  \else % if not pdftex
    \usepackage[default]{sourcesanspro}
  \usepackage{sourcecodepro}

  % XeLaTeX specific adjustments for straight quotes: https://tex.stackexchange.com/a/354887
  % This issue is already fixed (see https://github.com/silkeh/latex-sourcecodepro/pull/5) but the
  % fix is still unreleased.
  % TODO: Remove this workaround when the new version of sourcecodepro is released on CTAN.
  \ifxetex
    \makeatletter
    \defaultfontfeatures[\ttfamily]
      { Numbers   = \sourcecodepro@figurestyle,
        Scale     = \SourceCodePro@scale,
        Extension = .otf }
    \setmonofont
      [ UprightFont    = *-\sourcecodepro@regstyle,
        ItalicFont     = *-\sourcecodepro@regstyle It,
        BoldFont       = *-\sourcecodepro@boldstyle,
        BoldItalicFont = *-\sourcecodepro@boldstyle It ]
      {SourceCodePro}
    \makeatother
  \fi
  \fi

%
% heading color
%
\definecolor{heading-color}{RGB}{40,40,40}
\addtokomafont{section}{\color{heading-color}}
% When using the classes report, scrreprt, book,
% scrbook or memoir, uncomment the following line.
%\addtokomafont{chapter}{\color{heading-color}}

%
% variables for title, author and date
%
\usepackage{titling}
\title{}
\author{}
\date{}

%
% tables
%

%
% remove paragraph indention
%
\setlength{\parindent}{0pt}
\setlength{\parskip}{6pt plus 2pt minus 1pt}
\setlength{\emergencystretch}{3em}  % prevent overfull lines

%
%
% Listings
%
%


%
% header and footer
%
\usepackage{fancyhdr}

\fancypagestyle{eisvogel-header-footer}{
  \fancyhead{}
  \fancyfoot{}
  \lhead[]{}
  \chead[]{}
  \rhead[]{}
  \lfoot[\thepage]{}
  \cfoot[]{}
  \rfoot[]{\thepage}
  \renewcommand{\headrulewidth}{0.4pt}
  \renewcommand{\footrulewidth}{0.4pt}
}
\pagestyle{eisvogel-header-footer}

%%
%% end added
%%

\begin{document}

%%
%% begin titlepage
%%

%%
%% end titlepage
%%



{
\setcounter{tocdepth}{3}
\tableofcontents
}
\hypertarget{table-of-contents}{%
\subsubsection{Table of contents}\label{table-of-contents}}

\begin{itemize}
\tightlist
\item
  \protect\hyperlink{table-of-contents}{Table of contents}
\item
  \protect\hyperlink{javascript-basics}{Javascript Basics}

  \begin{itemize}
  \tightlist
  \item
    \protect\hyperlink{primitives-and-the-console}{Primitives and the
    Console}
  \item
    \protect\hyperlink{numbers}{Numbers}
  \item
    \protect\hyperlink{variables}{Variables}
  \item
    \protect\hyperlink{variable-types}{Variable types}
  \item
    \protect\hyperlink{booleans}{Booleans}
  \item
    \protect\hyperlink{variable-naming-and-conventions}{Variable naming
    and conventions}
  \end{itemize}
\item
  \protect\hyperlink{javascript-strings-and-more}{Javascript Strings and
  More}

  \begin{itemize}
  \tightlist
  \item
    \protect\hyperlink{strings}{Strings}
  \item
    \protect\hyperlink{indices-and-length}{Indices and length}
  \item
    \protect\hyperlink{string-methods}{String Methods}
  \item
    \protect\hyperlink{string-template-literals}{String Template
    Literals}
  \item
    \protect\hyperlink{undefined-and-null}{Undefined and Null}
  \item
    \protect\hyperlink{random-number-and-math-object}{Random number and
    Math Object}
  \end{itemize}
\item
  \protect\hyperlink{decision-making}{Decision Making}

  \begin{itemize}
  \tightlist
  \item
    \protect\hyperlink{comparison-operators}{Comparison Operators}
  \item
    \protect\hyperlink{equality}{Equality}
  \item
    \protect\hyperlink{console-alert-prompt}{Console, Alert Prompt}
  \item
    \protect\hyperlink{if-statement}{If Statement}
  \item
    \protect\hyperlink{truthy-and-falsy}{Truthy and Falsy}
  \item
    \protect\hyperlink{logical-expressions}{Logical Expressions}
  \item
    \protect\hyperlink{switch-statement}{Switch Statement}
  \end{itemize}
\item
  \protect\hyperlink{javascript-arrays}{Javascript Arrays}

  \begin{itemize}
  \tightlist
  \item
    \protect\hyperlink{introduction}{Introduction}
  \item
    \protect\hyperlink{modifying-arrays}{Modifying Arrays}
  \item
    \protect\hyperlink{array-methods}{Array Methods}
  \item
    \protect\hyperlink{reference-and-equality-types}{Reference and
    Equality Types}
  \item
    \protect\hyperlink{arrays-and-constant}{Arrays and Constant}
  \item
    \protect\hyperlink{multidimensional-arrays}{Multidimensional Arrays}
  \end{itemize}
\item
  \protect\hyperlink{javascript-objects}{Javascript Objects}

  \begin{itemize}
  \tightlist
  \item
    \protect\hyperlink{object-literals}{Object Literals}
  \item
    \protect\hyperlink{accessing-data-objects}{Accessing Data Objects}
  \item
    \protect\hyperlink{object-methods}{Object Methods}
  \item
    \protect\hyperlink{modifying-objects-imp}{Modifying Objects (Imp)}
  \item
    \protect\hyperlink{array-objects}{Array Objects}
  \end{itemize}
\item
  \protect\hyperlink{loops}{Loops}

  \begin{itemize}
  \tightlist
  \item
    \protect\hyperlink{for-loop}{For Loop}
  \item
    \protect\hyperlink{infinite-loops}{Infinite Loops}
  \item
    \protect\hyperlink{looping-arrays}{Looping Arrays}
  \item
    \protect\hyperlink{nested-loops}{Nested Loops}
  \item
    \protect\hyperlink{while-loop}{While Loop}
  \item
    \protect\hyperlink{for-of-loop}{For of Loop}
  \item
    \protect\hyperlink{iterating-objects}{Iterating Objects}
  \item
    \protect\hyperlink{to-do-exercise-refer-code}{To-Do Exercise (Refer
    Code)}
  \end{itemize}
\item
  \protect\hyperlink{functions}{Functions}

  \begin{itemize}
  \tightlist
  \item
    \protect\hyperlink{introduction-1}{Introduction}
  \item
    \protect\hyperlink{arguments}{Arguments}
  \item
    \protect\hyperlink{return-statement}{Return Statement}
  \end{itemize}
\item
  \protect\hyperlink{leveling-up-our-functions}{Leveling up our
  Functions}

  \begin{itemize}
  \tightlist
  \item
    \protect\hyperlink{scope}{Scope}
  \item
    \protect\hyperlink{block-scope}{Block Scope}
  \item
    \protect\hyperlink{declarations}{Declarations}
  \item
    \protect\hyperlink{lexical-scope}{Lexical Scope}
  \item
    \protect\hyperlink{scopes-final}{Scopes Final}
  \item
    \protect\hyperlink{function-expressions}{Function Expressions}
  \item
    \protect\hyperlink{higher-order-functions-important}{Higher Order
    Functions (Important)}
  \item
    \protect\hyperlink{returning-functions-important}{Returning
    Functions (Important)}
  \item
    \protect\hyperlink{methods}{Methods}
  \item
    \protect\hyperlink{this}{This}
  \item
    \protect\hyperlink{try-catch}{Try-Catch}
  \end{itemize}
\item
  \protect\hyperlink{arrays-and-callback-methods}{Arrays and Callback
  Methods}

  \begin{itemize}
  \tightlist
  \item
    \protect\hyperlink{for-each}{For each}
  \item
    \protect\hyperlink{map-method}{Map Method}
  \item
    \protect\hyperlink{intro-to-arrow-functions}{Intro to Arrow
    Functions}
  \item
    \protect\hyperlink{implicit-arrow-function}{Implicit Arrow Function}
  \item
    \protect\hyperlink{sorting-out-functions}{Sorting out Functions}
  \item
    \protect\hyperlink{settimeout-setinterval}{setTimeout, setInterval}
  \item
    \protect\hyperlink{filter-method}{Filter Method}
  \item
    \protect\hyperlink{some-and-every}{Some and Every}
  \item
    \protect\hyperlink{reduce}{Reduce}
  \item
    \protect\hyperlink{this-in-arrow-functions-refer-code}{This in Arrow
    Functions (Refer Code)}
  \end{itemize}
\end{itemize}

\hypertarget{javascript-basics}{%
\subsubsection{Javascript Basics}\label{javascript-basics}}

\hypertarget{primitives-and-the-console}{%
\paragraph{Primitives and the
Console}\label{primitives-and-the-console}}

\begin{itemize}
\tightlist
\item
  Basic Building Blocks in Javascript are

  \begin{enumerate}
  \def\labelenumi{\arabic{enumi}.}
  \tightlist
  \item
    Number
  \item
    String
  \item
    Boolean
  \item
    Null
  \item
    Undefined
  \end{enumerate}
\item
  Other two technical types

  \begin{enumerate}
  \def\labelenumi{\arabic{enumi}.}
  \tightlist
  \item
    Symbols
  \item
    Big Integers
  \item
    These two are way less commonly used.
  \end{enumerate}
\end{itemize}

\hypertarget{numbers}{%
\paragraph{Numbers}\label{numbers}}

\begin{itemize}
\tightlist
\item
  Comments

  \begin{enumerate}
  \def\labelenumi{\arabic{enumi}.}
  \tightlist
  \item
    Using doubleslash
  \item
    //This is a comment
  \end{enumerate}
\item
  Numbers

  \begin{enumerate}
  \def\labelenumi{\arabic{enumi}.}
  \tightlist
  \item
    Javascript has one number type. Some other languages have more than
    one.
  \item
    Positive Numbers
  \item
    Negative Numbers
  \item
    Whole Numbers(Integers)
  \item
    Decimal Numbers
  \end{enumerate}
\item
  Math operations

  \begin{enumerate}
  \def\labelenumi{\arabic{enumi}.}
  \tightlist
  \item
    Addition(+)
  \item
    Subtraction(-)
  \item
    Multiplication(*)
  \item
    Division(/)
  \item
    Modulo (Remainder operator - \%)
  \item
    Exponentiation (Power operator - **)
  \end{enumerate}
\item
  Not a Number(NaN)

  \begin{enumerate}
  \def\labelenumi{\arabic{enumi}.}
  \tightlist
  \item
    NaN is a numeric value that represents something that is not a
    number.
  \item
    \textbf{NaN in Javascript is considered as a number.}
  \item
    The reason NaN in Javascript is considered as a number is because it
    is technically a numeric data type whose value cannot be represented
    using actual numbers.
  \item
    Any operation along with NaN is NaN
  \item
    Example : 0/0 - NaN = NaN, 1 + NaN - NaN = NaN
  \item
    There are five different types of operations that return NaN:

    \begin{enumerate}
    \def\labelenumii{\arabic{enumii}.}
    \tightlist
    \item
      Number cannot be parsed (e.g.~parseInt(``blabla'') or
      Number(undefined))
    \item
      Math operation where the result is not a real number
      (e.g.~Math.sqrt(-1))
    \item
      Operand of an argument is NaN (e.g.~7 ** NaN)
    \item
      Indeterminate form (e.g.~0 * Infinity, or undefined + undefined)
    \item
      Any operation that involves a string and is not an addition
      operation (e.g.~``foo'' / 3)
    \end{enumerate}
  \end{enumerate}
\item
  Operators in Javascript.

  \begin{itemize}
  \tightlist
  \item
    The full list of the operators in Javascript can be found here.
    https://www.w3schools.com/js/js\_operators.asp
  \item
    In short.

    \begin{enumerate}
    \def\labelenumi{\arabic{enumi}.}
    \tightlist
    \item
      Arithmetic Operators (+,-,*,**,/,\%,++,--)
    \item
      Assignment Operators (=,+=,-=,*=,/=,*~*=,\%=)
    \item
      Comparison Operators (==, ===, !=, !==, \textgreater, \textless,
      \textgreater=, \textless=, ?)

      \begin{itemize}
      \tightlist
      \item
        Note that `===' and `!==' don't only check the value but they
        also look for the type of the variable.
      \end{itemize}
    \item
      Logical Operators (\&\&, \textbar\textbar, !)
    \item
      Type Operators (typeof, instanceof)
    \item
      Bitwise Operators (\&, \textbar, \textasciitilde, \^{},
      \textless\textless, \textgreater\textgreater,
      \textgreater\textgreater\textgreater)
    \end{enumerate}
  \end{itemize}
\item
  Escape Sequences

  \begin{enumerate}
  \def\labelenumi{\arabic{enumi}.}
  \tightlist
  \item
    (\textquotesingle) : Single quote
  \item
    (") : Double quote
  \item
    (\textbackslash) : Backslash
  \item
    (\n) : newline
  \item
    (\r) : carriage return
  \item
    (\t) : tab
  \item
    (\b) : word boundary
  \item
    (\f) : form feed
  \end{enumerate}
\end{itemize}

\hypertarget{variables}{%
\paragraph{Variables}\label{variables}}

\begin{itemize}
\tightlist
\item
  Variables are like labels for values
\item
  We can store a value and give it a name so that we can:

  \begin{itemize}
  \tightlist
  \item
    *Refer back to it later
  \item
    *Use that value to do\ldots{} stuff
  \item
    *Or change it later one
  \end{itemize}
\item
  Basic Syntax

  \begin{itemize}
  \tightlist
  \item
    let someName = value;
  \end{itemize}
\item
  Example

  \begin{itemize}
  \tightlist
  \item
    let year = 1985;
  \item
    console.log(year)
  \end{itemize}
\end{itemize}

\hypertarget{variable-types}{%
\paragraph{Variable types}\label{variable-types}}

\begin{itemize}
\tightlist
\item
  let
\item
  const

  \begin{enumerate}
  \def\labelenumi{\arabic{enumi}.}
  \tightlist
  \item
    const works just like let, except you cannot change the value.
  \item
    const variable does not allow to change the value
  \item
    We can use it to change the things we know won't change.
  \item
    Example : const pi = 3.14159;
  \end{enumerate}
\item
  var

  \begin{enumerate}
  \def\labelenumi{\arabic{enumi}.}
  \tightlist
  \item
    Before let and const, var was the only way of declaring variables.
    These days, there isn't really a reason to use it.
  \end{enumerate}
\end{itemize}

\hypertarget{booleans}{%
\paragraph{Booleans}\label{booleans}}

\begin{itemize}
\tightlist
\item
  Booleans are used to store True or False values.
\item
  Unlike Python , they are lowercase in Javascript.
\item
  You can change a number variable to boolean or any other type in
  javascript.
\end{itemize}

\hypertarget{variable-naming-and-conventions}{%
\paragraph{Variable naming and
conventions}\label{variable-naming-and-conventions}}

\begin{itemize}
\tightlist
\item
  In Javascript, identifiers are case sensitive, it can contain Unicode
  letters,\$,\_,digits but may not start with a digit.
\item
  Give variables camelCase names.
\item
  Give proper variable names (which can be understood.)
\end{itemize}

\hypertarget{javascript-strings-and-more}{%
\subsubsection{Javascript Strings and
More}\label{javascript-strings-and-more}}

\hypertarget{strings}{%
\paragraph{Strings}\label{strings}}

\begin{itemize}
\tightlist
\item
  Strings of Characters
\item
  Strings are another primitive type in Javascript.
\item
  \textbf{Strings in Javascript are immutable}
\item
  They represent text, and must be wrapped in quotes(single, double or
  triple).
\item
  Strings can also be empty - ``\,''
\item
  Strings can be added to each other i.e concatenated
\item
  Syntax -

  \begin{itemize}
  \tightlist
  \item
    ``String 1'' + ``String 2''
  \item
    result+=``String1'' (When you want to modify the string result by
    concatenating it with other string)
  \end{itemize}
\item
  Similarly, we can also concatenate a string and a number.
\item
  The number will be converted to string when concatenating it with a
  string.
\item
  Example - 1 + ``hi'' = 1hi(which is a string)
\item
  Also, we can overwrite a string to change its value.
\end{itemize}

\hypertarget{indices-and-length}{%
\paragraph{Indices and length}\label{indices-and-length}}

\begin{itemize}
\tightlist
\item
  Index
\end{itemize}

\begin{enumerate}
\def\labelenumi{\arabic{enumi}.}
\tightlist
\item
  Strings are indexed i.e each character has a corresponding index ( a
  positional number).
\item
  Index starts from 0.
\item
  We can access individual characters based on their index.
\item
  \textbf{But we cannot change the individual characters with the help
  of their index as strings are immutable in Javascript.}
\end{enumerate}

\begin{itemize}
\tightlist
\item
  Length
\end{itemize}

\begin{enumerate}
\def\labelenumi{\arabic{enumi}.}
\tightlist
\item
  Syntax - var.length
\item
  \textbf{length starts counting from 1 unlike index which starts
  counting from 0.}
\item
  \textbf{It is not a method but rather a property. So we do not add
  parentheses () unlike other methods.}
\end{enumerate}

\hypertarget{string-methods}{%
\paragraph{String Methods}\label{string-methods}}

\begin{itemize}
\tightlist
\item
  String Methods -
  https://developer.mozilla.org/en-US/docs/Web/JavaScript/Reference/Global\_Objects/String
\item
  Methods are built-in actions we can perform with individual strings.
\item
  They help us do things like

  \begin{itemize}
  \tightlist
  \item
    Searching within a string
  \item
    Replacing part of a string
  \item
    Changing the casing of a string
  \end{itemize}
\item
  Syntax \texttt{thing.method()}
\item
  Methods

  \begin{itemize}
  \tightlist
  \item
    We will identify methods based on destructive or non destructive.
    Destructive methods change the value of the original string without
    overwriting when the method is applied on it whereas non destructive
    method preserves the original string.
  \item
    Types of Methods

    \begin{itemize}
    \tightlist
    \item
      toUpperCase() - Changes all the characters of the string to Upper
      case (Non destructive method)
    \item
      toLowerCase() - Changes all the characters of the string to Lower
      casw (Non destructive method)
    \item
      trim() - Removes any whitespace at the beginning of the string or
      at the end (Non destructive method)
    \end{itemize}
  \end{itemize}
\item
  We can also chain methods together.
\end{itemize}

\hypertarget{string-template-literals}{%
\paragraph{String Template Literals}\label{string-template-literals}}

\begin{itemize}
\tightlist
\item
  Template literals are strings that allow embedded expressions, which
  will be evaluated and then turned into a resulting string.
\item
  Expressions can be embedded using \$\{expression\} where we can use
  variable or an expression.
\item
  `String is enclosed with the use of backticks.`
\end{itemize}

\hypertarget{undefined-and-null}{%
\paragraph{Undefined and Null}\label{undefined-and-null}}

\begin{itemize}
\tightlist
\item
  Null

  \begin{enumerate}
  \def\labelenumi{\arabic{enumi}.}
  \tightlist
  \item
    Intentional absence of any value is given as null.
  \item
    Must be assigned.
  \end{enumerate}
\item
  Undefined

  \begin{enumerate}
  \def\labelenumi{\arabic{enumi}.}
  \tightlist
  \item
    Variables that do not have an assigned value are undefined.
  \end{enumerate}
\end{itemize}

\hypertarget{random-number-and-math-object}{%
\paragraph{Random number and Math
Object}\label{random-number-and-math-object}}

\begin{itemize}
\tightlist
\item
  Math Object

  \begin{enumerate}
  \def\labelenumi{\arabic{enumi}.}
  \tightlist
  \item
    Contains properties and methods for mathematical constants and
    functions.
  \end{enumerate}
\end{itemize}

\hypertarget{decision-making}{%
\subsubsection{Decision Making}\label{decision-making}}

\hypertarget{comparison-operators}{%
\paragraph{Comparison Operators}\label{comparison-operators}}

\begin{itemize}
\tightlist
\item
  Comparisons

  \begin{enumerate}
  \def\labelenumi{\arabic{enumi}.}
  \tightlist
  \item
    `\textgreater{}' \ldots.. greater than
  \item
    `\textless{}' \ldots.. less than
  \item
    `\textgreater{}' \ldots.. greater than or equal to
  \item
    `\textless=' \ldots.. less than or equal to
  \item
    `==' \ldots.. equality
  \item
    `!=' \ldots.. not equal
  \item
    `===' \ldots.. strict equality
  \item
    `!==' \ldots.. strict inequality
  \end{enumerate}
\item
  Comparisons are usually used for numbers however we can also use it
  for strings where we compare the unicode values of the strings.
\item
  All of the comparison operators return a boolean true or false value.
\item
  Example
  \texttt{JS\ \ \ \ \ function\ isEqual(a,b)\ \{\ \ \ \ \ \ \ \ \ if\ (a\ ===\ b)\ \{\ \ \ \ \ \ \ \ \ \ \ \ \ return\ true;\ \ \ \ \ \ \ \ \ \}else\ \{\ \ \ \ \ \ \ \ \ \ \ \ \ return\ false;\ \ \ \ \ \ \ \ \ \}\ \ \ \ \ \}}
\item
  The better way to write this is:
  \texttt{JS\ \ \ \ \ function\ isEqual(a,b)\ \{\ \ \ \ \ \ \ \ \ return\ a\ ===\ b;\ \ \ \ \ \}}
\item
  This helps only in case of booleans as the comparison operators return
  true or false
\end{itemize}

\hypertarget{equality}{%
\paragraph{Equality}\label{equality}}

\begin{verbatim}
```JS
a = 14
b = "14"

console.log(a == b) //true
console.log(7 == "7") //true
console.log(0 == "") //true
console.log(0 == false) //true
console.log(null == undefined) //true
console.log("b" == "c") //false
console.log(true == false) //false

//In Javascript 
//Number("") == 0
//Number("923") = 923
//Number("923 123") = NaN
// Number("jskfdjd") = NaN
```
\end{verbatim}

\hypertarget{console-alert-prompt}{%
\paragraph{Console, Alert Prompt}\label{console-alert-prompt}}

\begin{itemize}
\tightlist
\item
  console.log()

  \begin{itemize}
  \tightlist
  \item
    prints arguments to the console.
  \end{itemize}
\item
  alert()

  \begin{itemize}
  \tightlist
  \item
    alerts a message in the browser or DOM
  \end{itemize}
\item
  prompt()

  \begin{itemize}
  \tightlist
  \item
    prompts the user asking for value in the browser window.
  \end{itemize}
\end{itemize}

\hypertarget{if-statement}{%
\paragraph{If Statement}\label{if-statement}}

\begin{itemize}
\tightlist
\item
  Conditional statements is making decisions with our code.
\item
  if statement only runs code inside the curly braces if the condition
  is true.
\item
  If the condition is false, nothing happens or either true elif or else
  condition is executed.
\item
  Syntax:
  \texttt{JS\ \ \ \ \ if(condition)\{\ \ \ \ \ \ \ \ \ code\ \ \ \ \ \}\ \ \ \ \ else\ if(condition)\{\ \ \ \ \ \ \ \ \ code\ \ \ \ \ \}\ \ \ \ \ else\{\ \ \ \ \ \ \ \ \ code\ \ \ \ \ \}}
\item
  Check code too for better understanding.
\end{itemize}

\hypertarget{truthy-and-falsy}{%
\paragraph{Truthy and Falsy}\label{truthy-and-falsy}}

\begin{itemize}
\tightlist
\item
  All JS Values have an inherent truthyness or falsyness around them.
\item
  \textbf{Falsy Values}

  \begin{enumerate}
  \def\labelenumi{\arabic{enumi}.}
  \tightlist
  \item
    false
  \item
    0 3.''\,``(empty string) 4.null 5.undefined 6.NaN
  \end{enumerate}
\item
  Everything else is truthy
\end{itemize}

\hypertarget{logical-expressions}{%
\paragraph{Logical Expressions}\label{logical-expressions}}

\begin{itemize}
\tightlist
\item
  Logical operators are the ones that allow us to combine different
  expressions.
\item
  So we can combine more than one piece of logic together to form one
  larger piece of logic.
\item
  Logical operators

  \begin{enumerate}
  \def\labelenumi{\arabic{enumi}.}
  \tightlist
  \item
    AND (\&\&) Both sides must be true for entire thing to be true.
  \item
    OR (\textbar\textbar) If one side is true, the entire thing is true.
  \item
    NOT (!) !expression returns true if expression is false and false if
    true.
  \end{enumerate}
\item
  \textbf{AND has precedence over OR. You can use parentheses to filter
  the }precendence**.
\end{itemize}

\hypertarget{switch-statement}{%
\paragraph{Switch Statement}\label{switch-statement}}

\begin{itemize}
\tightlist
\item
  The switch statement is another control-flow statement that can
  replace multiple if statements.
\item
  Syntax
  \texttt{JS\ \ \ \ \ switch(variable\ or\ number)\{\ \ \ \ \ \ \ \ \ case\ 1:\ \ \ \ \ \ \ \ \ \ \ \ \ //code;\ \ \ \ \ \ \ \ \ \ \ \ \ break;\ \ \ \ \ \ \ \ \ case\ 2:\ \ \ \ \ \ \ \ \ \ \ \ \ //code;\ \ \ \ \ \ \ \ \ \ \ \ \ break;\ \ \ \ \ \ \ \ \ case\ 3:\ \ \ \ \ \ \ \ \ \ \ \ \ //code;\ \ \ \ \ \ \ \ \ \ \ \ \ break;\ \ \ \ \ \ \ \ \ .\ \ \ \ \ \ \ \ \ .\ \ \ \ \ \ \ \ \ .\ \ \ \ \ \ \ \ \ case\ n:\ \ \ \ \ \ \ \ \ \ \ \ \ //code;\ \ \ \ \ \ \ \ \ \ \ \ \ break;\ \ \ \ \ \ \ \ \ default:\ \ \ \ \ \ \ \ \ \ \ \ \ //code;\ \ \ \ \ \ \ \ \ \}}
\item
  Working

  \begin{enumerate}
  \def\labelenumi{\arabic{enumi}.}
  \tightlist
  \item
    So switch detects the number or variable passed into it.
  \item
    It keeps on executing code ahead of it until the end of the loop or
    break .
  \item
    So if there are 7 cases without break and the parameter if 5, it
    will execute 5,6 and 7th cases.
  \item
    That's why break is necessary.
  \end{enumerate}
\item
  Example
  \texttt{JS\ \ \ \ \ function\ sequentialSizes(val)\ \{\ \ \ \ \ \ \ \ \ var\ answer\ =\ "";\ \ \ \ \ \ \ \ \ switch(val)\{\ \ \ \ \ \ \ \ \ \ \ \ \ case\ 1:\ \ \ \ \ \ \ \ \ \ \ \ \ case\ 2:\ \ \ \ \ \ \ \ \ \ \ \ \ case\ 3:\ \ \ \ \ \ \ \ \ \ \ \ \ answer\ =\ "Low";\ \ \ \ \ \ \ \ \ \ \ \ \ break;\ \ \ \ \ \ \ \ \ \ \ \ \ case\ 4:\ \ \ \ \ \ \ \ \ \ \ \ \ case\ 5:\ \ \ \ \ \ \ \ \ \ \ \ \ case\ 6:\ \ \ \ \ \ \ \ \ \ \ \ \ answer\ =\ "Mid";\ \ \ \ \ \ \ \ \ \ \ \ \ break;\ \ \ \ \ \ \ \ \ \ \ \ \ case\ 7:\ \ \ \ \ \ \ \ \ \ \ \ \ case\ 8:\ \ \ \ \ \ \ \ \ \ \ \ \ case\ 9:\ \ \ \ \ \ \ \ \ \ \ \ \ answer\ =\ "High";\ \ \ \ \ \ \ \ \ \ \ \ \ break;\ \ \ \ \ \ \ \ \ \}\ \ \ \ \ \ \ \ \ return\ answer;\ \ \ \ \ \}\ \ \ \ \ sequentialSizes(1);}
\item
  In the above example, answer will be Low for cases 1,2 and 3, Mid for
  cases 4,5 and 6, High for cases 7,8 and 9.
\end{itemize}

\hypertarget{javascript-arrays}{%
\subsubsection{Javascript Arrays}\label{javascript-arrays}}

\hypertarget{introduction}{%
\paragraph{Introduction}\label{introduction}}

\begin{itemize}
\tightlist
\item
  Ordered collection of values

  \begin{enumerate}
  \def\labelenumi{\arabic{enumi}.}
  \tightlist
  \item
    List of comments on IG Post.
  \item
    Collection of levels in a game.
  \item
    Songs in a playlist.
  \end{enumerate}
\item
  Creating Arrays.

  \begin{enumerate}
  \def\labelenumi{\arabic{enumi}.}
  \tightlist
  \item
    To make an empty array \texttt{let\ students\ =\ {[}{]};}
  \item
    An array of strings
    \texttt{let\ colors\ =\ {[}\textquotesingle{}red\textquotesingle{},\textquotesingle{}orange\textquotesingle{},\textquotesingle{}yellow\textquotesingle{}{]};}
  \item
    An array of numbers
    \texttt{let\ lottoNums\ =\ {[}19,22,56,12,51{]};}
  \item
    A mixed array
    \texttt{let\ stuff\ =\ {[}true,\ 68,\ \textquotesingle{}cat\textquotesingle{},\ null,\ undefined{]}}
  \item
    \textbf{Javascript can also create a mixed array unlike C++.}
  \end{enumerate}
\item
  Array Random access.

  \begin{enumerate}
  \def\labelenumi{\arabic{enumi}.}
  \tightlist
  \item
    Just like strings, arrays are also indexed. Each element has a
    corresponding index where the counting starts at 0.
  \item
    Example \texttt{colors{[}0{]}\ //red} \texttt{colors.length\ //3}
  \item
    If index is not present in the array, you get undefined.
  \item
    We can also chain indexes \texttt{colors{[}0{]}{[}0{]}\ //r}
  \end{enumerate}
\end{itemize}

\hypertarget{modifying-arrays}{%
\paragraph{Modifying Arrays}\label{modifying-arrays}}

\begin{itemize}
\tightlist
\item
  \textbf{Arrays can be modified unlike strings which cannot be modified
  but needs to be overwritten.}
\item
  Example
  \texttt{JS\ \ \ \ \ let\ colors\ =\ {[}\textquotesingle{}rad\textquotesingle{},\textquotesingle{}orange\textquotesingle{},\textquotesingle{}blue\textquotesingle{},\textquotesingle{}white\textquotesingle{},\textquotesingle{}black\textquotesingle{}{]};\ \ \ \ \ colors{[}0{]}=\ "red";}
\end{itemize}

\hypertarget{array-methods}{%
\paragraph{Array Methods}\label{array-methods}}

\begin{itemize}
\tightlist
\item
  Resources

  \begin{enumerate}
  \def\labelenumi{\arabic{enumi}.}
  \tightlist
  \item
    MDN Arrays -
    https://developer.mozilla.org/en-US/docs/Web/JavaScript/Reference/Global\_Objects/Array
  \item
    Slice, Splice, Split -
    https://www.freecodecamp.org/news/lets-clear-up-the-confusion-around-the-slice-splice-split-methods-in-javascript-8ba3266c29ae/
  \end{enumerate}
\item
  \textbf{Array Methods}

  \begin{enumerate}
  \def\labelenumi{\arabic{enumi}.}
  \tightlist
  \item
    Push - add to end (destructive method)
  \item
    Pop - remove from end (destructive method)
  \item
    Shift - remove from start (destructive method)
  \item
    Unshift - add to start (destructive method)
  \item
    concat - merge arrays(non-destructive for original arrays)
  \item
    includes - look for a value (false if not present)
  \item
    indexOf - returns the index of the array (-1 if not present and only
    returns first occurrence)
  \item
    join - creates a string from an array
  \item
    reverse - reverses an array (destructive method)
  \item
    slice - copies a portion of an array

    \begin{enumerate}
    \def\labelenumii{\arabic{enumii}.}
    \tightlist
    \item
      takes two optional parameters
    \item
      same as strings, index starts from 0 and goes till n-1.
    \item
      For negative index it will start from end of the index and assume
      that the index will start from 1 from the end.
    \end{enumerate}
  \item
    splice - removes / replaces elements

    \begin{enumerate}
    \def\labelenumii{\arabic{enumii}.}
    \tightlist
    \item
      Syntax array.(start\_index, delete\_count, items)
    \end{enumerate}
  \item
    sort - sorts an array

    \begin{enumerate}
    \def\labelenumii{\arabic{enumii}.}
    \tightlist
    \item
      Compares everything to string and then compares their UTF values
    \end{enumerate}
  \end{enumerate}
\end{itemize}

\hypertarget{reference-and-equality-types}{%
\paragraph{Reference and Equality
Types}\label{reference-and-equality-types}}

\begin{itemize}
\tightlist
\item
  \texttt{{[}1{]}\ ===\ {[}1{]}\ or\ {[}1{]}=={[}1{]}\ or\ {[}\textquotesingle{}\textquotesingle{}{]}=={[}\textquotesingle{}\textquotesingle{}{]}}

  \begin{enumerate}
  \def\labelenumi{\arabic{enumi}.}
  \tightlist
  \item
    This will give false as in arrays we are not comparing the content.
  \item
    Javascript doesn't care what's inside, atleast with arrays.
  \item
    \textbf{What it compares instead are the references in memory.}
  \item
    Let luckyNum = 87; Certain amount of space is allocated in memory
    for this number.
  \item
    For arrays that's not the case, you can have many many elements in
    an array where it stores lots of space.
  \item
    luckyNums = {[}1,2,3{]}
  \item
    Instead of associating luckyNums with all elements in luckyNums
    array, what Javascript does is it stores a reference which
    corresponds to array.
  \item
    So if you make another {[}1,2,3{]}, it is actually a different
    array. That is a new array in memory is created.
  \item
    \textbf{Refer the reference code as well.}
  \item
    Check the below code
  \end{enumerate}

\begin{Shaded}
\begin{Highlighting}[]
\KeywordTok{let}\NormalTok{ nums }\OperatorTok{=}\NormalTok{ [}\DecValTok{1}\OperatorTok{,}\DecValTok{2}\OperatorTok{,}\DecValTok{3}\NormalTok{]}
\NormalTok{numsCopy }\OperatorTok{=}\NormalTok{ nums}
\BuiltInTok{console}\OperatorTok{.}\FunctionTok{log}\NormalTok{(nums)}
\BuiltInTok{console}\OperatorTok{.}\FunctionTok{log}\NormalTok{(numsCopy)}
\NormalTok{nums}\OperatorTok{.}\FunctionTok{push}\NormalTok{(}\DecValTok{4}\NormalTok{)}
\BuiltInTok{console}\OperatorTok{.}\FunctionTok{log}\NormalTok{(nums)}
\BuiltInTok{console}\OperatorTok{.}\FunctionTok{log}\NormalTok{(numsCopy)}
\end{Highlighting}
\end{Shaded}

  \begin{enumerate}
  \def\labelenumi{\arabic{enumi}.}
  \setcounter{enumi}{10}
  \tightlist
  \item
    In the above code, numsCopy doesn't create a different copy of nums,
    however, it references to the same array as nums. So any changes
    made to numsCopy will reflect to nums and vice-versa.
  \end{enumerate}
\end{itemize}

\hypertarget{arrays-and-constant}{%
\paragraph{Arrays and Constant}\label{arrays-and-constant}}

\begin{itemize}
\tightlist
\item
  For strings we cannot change the value of a const
\item
  For arrays, the shell remains the same but the content can change
\item
  \textbf{Updating the content doesn't change the address of the array,
  the reference.}
\item
  \textbf{Example - As long as egg carton remains change you can put
  different eggs in it and change.}
\item
  As soon as you try to change it to new reference, you get an error.
\item
  \textbf{In short, the difference between strings and arrays is that
  the strings are immutable and the arrays are mutable. However, in case
  of const, we cannot re-assign values to both of them.}
\end{itemize}

\hypertarget{multidimensional-arrays}{%
\paragraph{Multidimensional Arrays}\label{multidimensional-arrays}}

\begin{itemize}
\tightlist
\item
  Also known as Nested Arrays.
\item
  We can store arrays inside other arrays.
\item
  Refer code.
\end{itemize}

\hypertarget{javascript-objects}{%
\subsubsection{Javascript Objects}\label{javascript-objects}}

\hypertarget{object-literals}{%
\paragraph{Object Literals}\label{object-literals}}

\begin{itemize}
\tightlist
\item
  Objects are collections of properties.
\item
  Properties are a key-value pair.
\item
  Rather than accessing data using an index, we use custom keys.
\item
  It consists of property key + value
\item
  It is similar to the dictionaries in python.
\item
  Example:
  \texttt{JS\ \ \ \ \ const\ fitBitData\ =\ \{\ \ \ \ \ \ \ \ \ totalSteps\ :\ 308727,\ \ \ \ \ \ \ \ \ totalMiles\ :\ 211.7\ \ \ \ \ \ \ \ \ avgCalorieBurn\ :\ 5755,\ \ \ \ \ \ \ \ \ workoutsThisWeek\ :\ \textquotesingle{}5\ of\ 7\textquotesingle{}\ \ \ \ \ \ \ \ \ avgGoodSleep\ :\ \textquotesingle{}2:13\textquotesingle{}\ \ \ \ \ \}}
\item
  All valid keys are converted to strings in objects.(Except for symbols
  - which is uncommon)
\end{itemize}

\hypertarget{accessing-data-objects}{%
\paragraph{Accessing Data Objects}\label{accessing-data-objects}}

\begin{itemize}
\tightlist
\item
  Let us start with an example
\end{itemize}

\begin{Shaded}
\begin{Highlighting}[]
\NormalTok{person }\OperatorTok{=}\NormalTok{ \{}
    \DataTypeTok{firstName} \OperatorTok{:} \StringTok{"Yash"}\OperatorTok{,}
    \DataTypeTok{lastName} \OperatorTok{:} \StringTok{"Thakkar"}
\NormalTok{\}}
\end{Highlighting}
\end{Shaded}

\begin{itemize}
\tightlist
\item
  In order to access firstName
\end{itemize}

\begin{Shaded}
\begin{Highlighting}[]
\NormalTok{person[}\StringTok{"firstName"}\NormalTok{]}
\NormalTok{person}\OperatorTok{.}\AttributeTok{firstName}
\end{Highlighting}
\end{Shaded}

\begin{itemize}
\tightlist
\item
  {[}{]}

  \begin{enumerate}
  \def\labelenumi{\arabic{enumi}.}
  \tightlist
  \item
    For a key which is a string, it expects it to be a variable name in
    {[}{]}, therefore you need to wrap it up with double quotes.
  \item
    However, you can let numeric, boolean keys in double quotes as well
    as without it
  \end{enumerate}
\item
  dot access

  \begin{enumerate}
  \def\labelenumi{\arabic{enumi}.}
  \tightlist
  \item
    Whereas in dot method, you shouldn't wrap it in double quotes.
  \item
    Dot method doesn't work for numbers.
  \item
    It doesn't support variables as well
  \end{enumerate}
\end{itemize}

\hypertarget{object-methods}{%
\paragraph{Object Methods}\label{object-methods}}

\begin{Shaded}
\begin{Highlighting}[]
\KeywordTok{const}\NormalTok{ counterStrikeRating }\OperatorTok{=}\NormalTok{ \{}
    \DataTypeTok{Morris} \OperatorTok{:} \DecValTok{85}\OperatorTok{,}
    \DataTypeTok{Cooper} \OperatorTok{:} \DecValTok{80}\OperatorTok{,}
    \DataTypeTok{Ben} \OperatorTok{:} \DecValTok{88}\OperatorTok{,}
    \DataTypeTok{Sam} \OperatorTok{:} \DecValTok{95}\OperatorTok{,}
    \DataTypeTok{Maverick} \OperatorTok{:} \DecValTok{99}\OperatorTok{,}
    \DataTypeTok{Yash} \OperatorTok{:} \DecValTok{100}\OperatorTok{,}
    \DataTypeTok{Ace} \OperatorTok{:} \DecValTok{96}\OperatorTok{,}
    \DataTypeTok{Tex} \OperatorTok{:} \DecValTok{96}\OperatorTok{,}
    \DataTypeTok{Steel} \OperatorTok{:} \DecValTok{97}
\NormalTok{\}}
\end{Highlighting}
\end{Shaded}

\begin{enumerate}
\def\labelenumi{\arabic{enumi}.}
\tightlist
\item
  Object.keys(counterStrikeRating)
\item
  Object.values(counterStrikeRating)
\item
  Object.entries(counterStrikeRating)
\item
  counterStrikeRating.hasOwnProperty(propertyname)
\item
  Example \texttt{counterStrikeRating.hasOwnProperty("Morris");}
\end{enumerate}

\hypertarget{modifying-objects-imp}{%
\paragraph{Modifying Objects (Imp)}\label{modifying-objects-imp}}

\begin{itemize}
\tightlist
\item
  Check out the \texttt{03\_imp\_modifying\_objects.js} code.
\end{itemize}

\hypertarget{array-objects}{%
\paragraph{Array Objects}\label{array-objects}}

\begin{itemize}
\item
  Example
  \texttt{JS\ \ \ \ \ const\ student\ =\ \{\ \ \ \ \ \ \ \ \ firstName\ :\ \textquotesingle{}Yash\textquotesingle{},\ \ \ \ \ \ \ \ \ lastName\ :\ \textquotesingle{}Thakkar\textquotesingle{},\ \ \ \ \ \ \ \ \ strengths\ :\ {[}\textquotesingle{}Music\textquotesingle{},\ \textquotesingle{}Programming\textquotesingle{}{]},\ \ \ \ \ \ \ \ \ exams\ :\ \{\ \ \ \ \ \ \ \ \ \ \ \ \ midterms\ :\ 95,\ \ \ \ \ \ \ \ \ \ \ \ \ final\ :\ 100\ \ \ \ \ \ \ \ \ \}\ \ \ \ \ \}}
\item
  Example
  \texttt{JS\ \ \ \ \ const\ shoppingCart\ =\ {[}\ \ \ \ \ \ \ \ \ \{\ \ \ \ \ \ \ \ \ \ \ \ \ product\ :\ "POCO\ C3",\ \ \ \ \ \ \ \ \ \ \ \ \ price\ :\ 7000,\ \ \ \ \ \ \ \ \ \ \ \ \ quatity:\ 10\ \ \ \ \ \ \ \ \ \},\ \ \ \ \ \ \ \ \ \{\ \ \ \ \ \ \ \ \ \ \ \ \ product\ :\ "POCO\ M2",\ \ \ \ \ \ \ \ \ \ \ \ \ price\ :\ 9000,\ \ \ \ \ \ \ \ \ \ \ \ \ quatity:\ 5\ \ \ \ \ \ \ \ \ \},\ \ \ \ \ \ \ \ \ \{\ \ \ \ \ \ \ \ \ \ \ \ \ product\ :\ "Asus\ Max\ Pro\ M3",\ \ \ \ \ \ \ \ \ \ \ \ \ price\ :\ 9000,\ \ \ \ \ \ \ \ \ \ \ \ \ quantity:\ 1\ \ \ \ \ \ \ \ \ \}\ \ \ \ \ {]}}
\end{itemize}

\hypertarget{loops}{%
\subsubsection{Loops}\label{loops}}

\hypertarget{for-loop}{%
\paragraph{For Loop}\label{for-loop}}

\begin{itemize}
\tightlist
\item
  Loops allow us to repeat code
\item
  Sum all numbers in an array.
\item
  There are multiple types:

  \begin{enumerate}
  \def\labelenumi{\arabic{enumi}.}
  \tightlist
  \item
    for loop
  \item
    while loop
  \item
    for\ldots of loop
  \item
    for\ldots in loop
  \end{enumerate}
\item
  For Loop Syntax
  \texttt{JS\ \ \ \ \ for(\ \ \ \ \ \ \ \ \ {[}initialExpression{]};\ \ \ \ \ \ \ \ \ {[}condition{]};\ \ \ \ \ \ \ \ \ {[}incrementExpression{]}\ \ \ \ \ )}
\item
  Example
  \texttt{JS\ \ \ \ \ for\ (let\ i\ =\ 1;\ i\textless{}=10;\ i++)\{\ \ \ \ \ \ \ \ \ console.log(i)\ \ \ \ \ \}}
\item
  The loop keeps on executing until the condition is false
\item
  Hence

  \begin{enumerate}
  \def\labelenumi{\arabic{enumi}.}
  \tightlist
  \item
    The first part is the initialization part
  \item
    The second part is the condition part, if it is true, the loop will
    get executed, else it will not
  \item
    The third part is the incrementing part, so that the condition can
    be false and the loop comes to an end.
  \end{enumerate}
\end{itemize}

\hypertarget{infinite-loops}{%
\paragraph{Infinite Loops}\label{infinite-loops}}

\begin{itemize}
\tightlist
\item
  Infinite loops should be avoidable.
\item
  Loops which keep on going as there is no false condition are known as
  infinite loops and they keep on running.
\item
  They take up computer memory.
\end{itemize}

\hypertarget{looping-arrays}{%
\paragraph{Looping Arrays}\label{looping-arrays}}

\begin{itemize}
\tightlist
\item
  To loop over an array, start at index 0 and continue looping to until
  last index (length-1)
\item
  Example
  \texttt{JS\ \ \ \ \ const\ animals\ =\ {[}\textquotesingle{}lions\textquotesingle{},\textquotesingle{}tigers\textquotesingle{},\textquotesingle{}bears\textquotesingle{}{]};\ \ \ \ \ for(let\ i\ =\ 0;i\textless{}animals.length;i++)\{\ \ \ \ \ \ \ \ \ console.log(i,animals{[}i{]});\ \ \ \ \ \}}
\end{itemize}

\hypertarget{nested-loops}{%
\paragraph{Nested Loops}\label{nested-loops}}

\begin{itemize}
\tightlist
\item
  Example
  \texttt{JS\ \ \ \ \ let\ str\ =\ \textquotesingle{}LOL\textquotesingle{};\ \ \ \ \ for\ (let\ i\ =\ 0\ ;i\textless{}=4;\ i++)\{\ \ \ \ \ \ \ \ \ console.log("Outer",i)\ \ \ \ \ \ \ \ \ for\ (let\ j\ =\ 0;\ j\textless{}str.length;j++)\{\ \ \ \ \ \ \ \ \ \ \ \ \ console.log("Inner:",str{[}j{]})\ \ \ \ \ \ \ \ \ \}\ \ \ \ \ \}}
\item
  For every single iteration of the outer loop, let's say it runs five
  times, the inner loop is going to have it's own full cycle
\item
  while i is 1, we have an entire nested loop which runs for j starting
  at 0 upto 2. So we end up with 3 iterations of j all when we are at
  1st iteration of i, then we move up where i is 2 and the whole process
  repeats
\item
  This is how it works
  \texttt{0\ \ \ \ \ \ \ \ \ 0\ \ \ \ \ \ \ \ \ 1\ \ \ \ \ \ \ \ \ 2\ \ \ \ \ 1\ \ \ \ \ \ \ \ \ 0\ \ \ \ \ \ \ \ \ 1\ \ \ \ \ \ \ \ \ 2\ \ \ \ \ 2\ \ \ \ \ \ \ \ \ 0\ \ \ \ \ \ \ \ \ 1\ \ \ \ \ \ \ \ \ 2\ \ \ \ \ 3\ \ \ \ \ \ \ \ \ 0\ \ \ \ \ \ \ \ \ 1\ \ \ \ \ \ \ \ \ 2\ \ \ \ \ 4\ \ \ \ \ \ \ \ \ 0\ \ \ \ \ \ \ \ \ 1\ \ \ \ \ \ \ \ \ 2}
\end{itemize}

\hypertarget{while-loop}{%
\paragraph{While Loop}\label{while-loop}}

\begin{itemize}
\tightlist
\item
  while loops continue running as long as the test condition is true.
\item
  Example let num = 0 while (num \textless{} 10)\{ console.log(num);
  num++; \}
\item
  break keyword

  \begin{enumerate}
  \def\labelenumi{\arabic{enumi}.}
  \tightlist
  \item
    break keyword breaks out of the while loop.
  \item
    It is especially useful in case when we want to loop a statement
    again and again except for a certain condition.
  \item
    We can use break keyword for that particular condition.
  \end{enumerate}
\end{itemize}

\hypertarget{for-of-loop}{%
\paragraph{For of Loop}\label{for-of-loop}}

\begin{itemize}
\tightlist
\item
  Syntax
  \texttt{JS\ \ \ \ \ for\ (variable\ of\ iterable)\{\ \ \ \ \ \ \ \ \ statements;\ \ \ \ \ \}}
\item
  for of loop iterates through the whole string and array
\end{itemize}

\hypertarget{iterating-objects}{%
\paragraph{Iterating Objects}\label{iterating-objects}}

\begin{itemize}
\tightlist
\item
  for in loop is used to iterate over the objects.
\item
  Its Syntax is
  \texttt{JS\ \ \ \ \ for\ (variable\ in\ iterable)\{\ \ \ \ \ \ \ \ \ statements;\ \ \ \ \ \}}
\item
  using for in you can get the key of the key-value pair.
\item
  in order to get the value, you can write
  \texttt{console.log(iterable{[}variable{]})}
\item
  Other special methods for objects

  \begin{enumerate}
  \def\labelenumi{\arabic{enumi}.}
  \tightlist
  \item
    Object.keys(Object\_variable) //Returns array of keys
  \item
    Object.values(Object\_variable) //Returns array of values
  \item
    Object.entries(Object\_variable) //To get nested array of key-value
    pairs
  \end{enumerate}
\end{itemize}

\hypertarget{to-do-exercise-refer-code}{%
\paragraph{To-Do Exercise (Refer
Code)}\label{to-do-exercise-refer-code}}

\begin{itemize}
\tightlist
\item
  This is a basic to do app.
\item
  We will keep asking for choice until the user decides to quit.
\item
  We have created an empty array in the beginning.
\item
  If user enters new, and enters a todo. The todo is pushed into the
  array.
\item
  If the user wants to view his todo, then we will parse through the
  todo array.
\item
  i+1 is used to show first todo index as 1.
\item
  if choice is delete, we will ask user for the index of todo to delete
  and parse it to int as prompt accepts input as string.
\item
  Then we will check if the input entered by the user in prompt is
  actually a number or not.
\item
  If it is NaN(that is isNaN is true), then the first condition will get
  false and the else loop will be executed and valid index will be
  asked.
\item
  If it is not NaN, then the first condition will get true and the if
  loop will be executed.
\item
  In the if loop, we will make use of splice to remove the element.
\item
  remove-1 is used as the index is shown incremented as discussed in
  point 6.
\item
  Then one element is removed at that index and the deleted index and
  the name of the deleted todo is shown on window.
\end{itemize}

\hypertarget{functions}{%
\subsubsection{Functions}\label{functions}}

\hypertarget{introduction-1}{%
\paragraph{Introduction}\label{introduction-1}}

\begin{itemize}
\tightlist
\item
  Functions are reusable procedures
\item
  Functions allow us to write reusable, modular code.
\item
  We define a ``chunk'' of code that we can execute at a later point.
\item
  We use them all the time.
\item
  It is a 2 step process

  \begin{enumerate}
  \def\labelenumi{\arabic{enumi}.}
  \tightlist
  \item
    We need to define the function
  \item
    We need to run the function
  \end{enumerate}
\item
  Defining a function
  \texttt{JS\ \ \ \ \ function\ funcName()\{\ \ \ \ \ \ \ \ \ //do\ something\ \ \ \ \ \}}
\item
  Always define the function before you use them.
\item
  Also, technically functions are objects behind the scenes.
\end{itemize}

\hypertarget{arguments}{%
\paragraph{Arguments}\label{arguments}}

\begin{itemize}
\tightlist
\item
  We can also write functions that accept inputs, called arguments !
\item
  We have seen this before

  \begin{enumerate}
  \def\labelenumi{\arabic{enumi}.}
  \tightlist
  \item
    ``hello''.indexOf(`h') //We pass `h' as argument here
  \item
    nums.push(5,6,7)
  \end{enumerate}
\item
  Example
  \texttt{JS\ \ \ \ \ function\ greet(person)\{\ //Parameter\ \ \ \ \ \ \ \ \ console.log(\textasciigrave{}Hi,\ \$\{person\}\ !\textasciigrave{});\ \ \ \ \ \}\ \ \ \ \ greet("Yash")\ //Arguments}
\item
  If an expected argument is not passed to the function, it's gonna have
  a value of undefined.
\item
  And if an extra argument is passed to the function, it accepts the
  first argument and discards the second.
\item
  This happens only in Javascript and not in other languages.
\item
  \textbf{Functions with Multiple Arguments}

  \begin{enumerate}
  \def\labelenumi{\arabic{enumi}.}
  \tightlist
  \item
    We can pass multiple arguments in a function.
  \item
    The arguments are matched to the parameters with respect to the
    order.. That is first parameter will hold the value of first
    argument, second of second argument and so on.
  \end{enumerate}
\end{itemize}

\hypertarget{return-statement}{%
\paragraph{Return Statement}\label{return-statement}}

\begin{itemize}
\tightlist
\item
  return keyword is used to capture the return value of a function to a
  variable.
\item
  The return statement ends function execution AND specifies the value
  to be returned by that function.
\item
  That is statements after the return keyword in a function are not
  executed.
\end{itemize}

\hypertarget{leveling-up-our-functions}{%
\subsubsection{Leveling up our
Functions}\label{leveling-up-our-functions}}

\hypertarget{scope}{%
\paragraph{Scope}\label{scope}}

\begin{itemize}
\tightlist
\item
  Scope - Variable ``visibility''
\item
  The location where a variable is defined dictates where we have access
  to that variable.
\item
  Variables we define inside a function are scoped to that function,
  therefore if we use a function variable outside a function, then we
  get an error.
\item
  Example
  \texttt{JS\ \ \ \ \ let\ msg="I\ am\ on\ water!"\ \ \ \ \ function\ helpMe()\{\ \ \ \ \ \ \ \ \ let\ msg\ =\ "I\ am\ on\ fire!";\ \ \ \ \ \ \ \ \ console.log(msg);\ \ \ \ \ \}\ \ \ \ \ \ \ \ console.log(msg);\ //\ I\ am\ on\ water!}
  Here the msg is scoped to the helpMe function.
\item
  If there is a variable defined by the same name in the function, in
  the function scope, then the closer reference will be used. That is in
  a function for console.log closer, if there is a variable with the
  same name as outside the function, the function's variable will be
  used, else the outer variable.
\item
  \textbf{In case of let, the variables defined inside the function are
  not usable outside it.}
\item
  Example
  \texttt{JS\ \ \ \ \ let\ bird\ =\ "mandarin\ duck"\ \ \ \ \ function\ birdWatch()\{\ \ \ \ \ \ \ \ \ let\ bird\ =\ "goldenpheasant";\ \ \ \ \ \ \ \ \ bird;\ //goldenpheasant\ \ \ \ \ \}\ \ \ \ \ bird;\ //mandarin\ duck}
\item
  Although, if you declare a variable outside the function, change it in
  function, call the function and then use it. Then you can get the
  changed value inside the function PROVIDED it is not declared inside
  the function.
\item
  This also means that a variable declared outside the function is
  available to the function too.
\item
  Variables which are defined outside of a function block have Global
  scope. This means they can be seen everywhere in your Javascript code.
\item
  Also in order to use it, you don't need use global keyword in a
  function unlike python where if you want to modify the same global
  variable, you need to specify it as global.
\end{itemize}

\hypertarget{block-scope}{%
\paragraph{Block Scope}\label{block-scope}}

\begin{itemize}
\tightlist
\item
  Variables defined in a block are just scoped to the block.
\item
  Blocks refers to anytime we see curly braces except for a function.
\item
  Example :
  \texttt{JS\ \ \ \ \ let\ radius\ =\ 8;\ \ \ \ \ if\ (radius\textgreater{}0)\{\ \ \ \ \ \ \ \ \ const\ PI\ =\ 3.14;\ \ \ \ \ \ \ \ \ let\ circ\ =\ 2*PI*radius;\ \ \ \ \ \}\ \ \ \ \ console.log(radius);\ //8\ \ \ \ \ console.log(PI);\ //Not\ Defined\ \ \ \ \ console.log(circ);\ //Not\ Defined}
\item
  Here, PI and circ are scoped to the block
\item
  Same as functions, if the variables are declared outside the block and
  later modified in the block and then used outside, the output will be
  changed value PROVIDED it is not declared inside the scope.
\item
  This also means that a variable declared outside the block is
  available to the block too.
\end{itemize}

\hypertarget{declarations}{%
\paragraph{Declarations}\label{declarations}}

\begin{itemize}
\tightlist
\item
  Using var keyword, variables are scoped to functions but they are not
  scoped to blocks.
\item
  That is, when var keyword is used inside a function, the variable is
  not available outside the function, however, in case of blocks when
  var is used, the variable is available outside the block as well.
\item
  let and const are block and function scoped.
\item
  When you have local and global variables with the same name , the
  local variable takes a precedence over the global variable.
\item
  It is usually recommended to avoid the use of var keyword.
\end{itemize}

\hypertarget{lexical-scope}{%
\paragraph{Lexical Scope}\label{lexical-scope}}

\begin{itemize}
\item
  An inner function nested inside of some parent function has access to
  the scope or the variables in the scope of that outer function.
\item
  Example ```JS function outer()\{ let hero = ``Rune King Thor'';

\begin{verbatim}
  function inner(){
      lt cryForHelp = `${hero}, please save me !`
      console.log(cryForHelp);
  }
  inner();
\end{verbatim}

  \} ```
\item
  If something exists in outer(), we have access to it in inner() but
  this doesn't work the either way.
\end{itemize}

\hypertarget{scopes-final}{%
\paragraph{Scopes Final}\label{scopes-final}}

\begin{itemize}
\tightlist
\item
  So, there are scopes such as function scope, block scope and lexical
  scope.
\item
  Function scope

  \begin{itemize}
  \tightlist
  \item
    Variables declared inside the function are scoped to the function.
  \item
    If two variables with same name are declared, then the one having
    closer reference will be used.
  \item
    We can declare a variable, \textbf{change} the variable in function,
    then call the function and then can use it in program outside of
    that particular function.
  \item
    Variables declared outside the function are available to the
    function but not vice-versa
  \end{itemize}
\item
  Block Scope

  \begin{itemize}
  \tightlist
  \item
    Just as functions, variables declared in the block are just scoped
    to the block.
  \item
    Blocks refer to anytime we see curly braces except for a function
    such as if, for, while, etc.
  \item
    Rest it has all properties same as functions except for one thing
    and that is var keyword.
  \item
    var keyword is not block-scoped and is just function scoped or
    globally-scoped.
  \item
    That is variables declared using var keyword in the blocks, is
    available to the global scope as well.
  \item
    However, it is recommended to avoid use of var keyword and just use
    let and const and hence, block and function scoped are usually same
    for variables declared with these two declarations.
  \end{itemize}
\item
  Lexical Scope

  \begin{itemize}
  \tightlist
  \item
    An inner function nested inside of an outer parent function has
    access to the variables in the scope of that outer function.
  \item
    However, it is not the same case vice-versa.
  \end{itemize}
\end{itemize}

\hypertarget{function-expressions}{%
\paragraph{Function Expressions}\label{function-expressions}}

\begin{itemize}
\tightlist
\item
  There is a different way of defining a function and it actually
  involves storing the function in a variable.
\item
  Example
  \texttt{JS\ \ \ \ \ const\ square\ =\ function(num)\{\ \ \ \ \ \ \ \ \ \ return\ num*num;\ \ \ \ \ \}\ \ \ \ \ square(7);}
\item
  The above is a function expression, the right side after the variable
  add creates a function.
\item
  This function is stored in a variable that is square.
\item
  It is same as storing variable, object, array or a number. It is the
  same exact concept.
\item
  It is used in same way as any other function.
\item
  Here square is not the name of the function but name of the variable.
  That is we are storing a function with no name inside of a variable.
\item
  Whereas in normal cases, we make a function with its own name. Both
  behave the exact same way.
\item
  The first case is just like storing a value in variable. Example

  \begin{itemize}
  \tightlist
  \item
    const PI = 3.14
  \item
    3.14 does not have a name but the variable(or container) does have a
    name.
  \end{itemize}
\item
  Functions are values in Javascript, we can store them, we can pass
  them around just like number, array, string. Javascript considers
  functions just like any other value
\item
  If you call the square as square, you'll be returned the entire
  function and to use it you have to call it by doing square() just like
  normal functions.
\end{itemize}

\hypertarget{higher-order-functions-important}{%
\paragraph{Higher Order Functions
(Important)}\label{higher-order-functions-important}}

\begin{itemize}
\item
  Higher order functions are functions that work with other functions so
  that they operate on/with other functions.
\item
  They can

  \begin{enumerate}
  \def\labelenumi{\arabic{enumi}.}
  \tightlist
  \item
    Accept other functions as arguments.
  \item
    Return a function.
  \end{enumerate}
\item
  Function as arguments example: ```JS function callTwice(func)\{
  func(); func(); \}

  function laugh()\{ console.log(``HAHAHAHAHAHA'') \} callTwice(laugh)
  ```
\item
  Here, a mistake not to be made is passing laugh as laugh().
\item
  Because what this will do is it will execute the laugh and output the
  result and then pass ``HAHAHA\ldots.'' as an argument.
\item
  But this is not what we want to do, what we want to do is pass through
  the value of the function i.e laugh so that inside of callTwice it can
  be executed.
\end{itemize}

\hypertarget{returning-functions-important}{%
\paragraph{Returning Functions
(Important)}\label{returning-functions-important}}

\begin{itemize}
\tightlist
\item
  Refer the Returning functions code.
\item
  If you call the makeMysteryfunc(), it will return the entire function.
\item
  You need to capture the entire return value.
\item
  Most important thing to remember is, if you are calling the main
  function, you need to save the function in a variable. And then call
  the variable with the inner function's parameters.
\end{itemize}

\hypertarget{methods}{%
\paragraph{Methods}\label{methods}}

\begin{itemize}
\tightlist
\item
  We can add functions as properties on objects.
\item
  We call them methods.
\item
  Example to create a method
  \texttt{JS\ \ \ \ \ const\ math\ =\ \{\ \ \ \ \ \ \ \ \ multiply\ :\ function(x,y)\{\ \ \ \ \ \ \ \ \ \ \ \ \ return\ x*y;\ \ \ \ \ \ \ \ \ \},\ \ \ \ \ \ \ \ \ divide\ :\ function(x,y)\{\ \ \ \ \ \ \ \ \ \ \ \ \ return\ x/y;\ \ \ \ \ \ \ \ \ \},\ \ \ \ \ \ \ \ \ square\ :\ function(x)\{\ \ \ \ \ \ \ \ \ \ \ \ \ return\ x*x;\ \ \ \ \ \ \ \ \ \}\ \ \ \ \ \};}
\item
  Some built-in methods are toUpperCase(), push(), indexOf()
\item
  Shorthand method to use Methods in Javascript
  \texttt{JS\ \ \ \ \ const\ math\ =\ \{\ \ \ \ \ \ \ \ \ blah\ :\ \textquotesingle{}Hi\ !\textquotesingle{},\ \ \ \ \ \ \ \ \ add(x,y)\{\ \ \ \ \ \ \ \ \ \ \ \ \ return\ x+y;\ \ \ \ \ \ \ \ \ \},\ \ \ \ \ \ \ \ \ multiply(x,y)\{\ \ \ \ \ \ \ \ \ \ \ \ \ return\ x*y;\ \ \ \ \ \ \ \ \ \}\ \ \ \ \ \}\ \ \ \ \ math.add(50,60)\ //110}
\end{itemize}

\hypertarget{this}{%
\paragraph{This}\label{this}}

\begin{itemize}
\tightlist
\item
  Use the keyword this to access other properties of the same object.
\item
  Example
  \texttt{JS\ \ \ \ \ const\ person\ =\ \{\ \ \ \ \ \ \ \ \ first\ :\ \textquotesingle{}Yash\textquotesingle{},\ \ \ \ \ \ \ \ \ last\ :\ \textquotesingle{}Thakkar\textquotesingle{},\ \ \ \ \ \ \ \ \ fullName()\{\ \ \ \ \ \ \ \ \ \ \ \ \ return\ \textasciigrave{}\$(this.first)\ \$(this.last)\textasciigrave{}\ \ \ \ \ \ \ \ \ \}\ \ \ \ \ \}\ \ \ \ \ person.fullName()\ //Yash\ Thakkar\ \ \ \ \ person.last\ =\ ""\ \ \ \ \ person.fullName()\ //Yash}
\item
  The value of `this' can change and it depends on the invocation
  context of the function it is used in.
\item
  That is it depends on how we call the function.
\item
  Usually when we use a method, like person.fullName(), the object
  before the dot(.) is invoked. So the value for the this.first would be
  key first of object person.
\item
  However when we do something like,
  \texttt{JS\ \ \ \ \ const\ fullName2\ =\ person.fullName;\ \ \ \ \ fullName2();}
\item
  The fullName2() does not refer to the person object.
\item
  The fullName2() references to the \textbf{window object which is the
  by default value as there is not dot(.) operator} before the
  fullName2() variable and hence it doesn't refer to any object.
\item
  When you create any function, it is added to the window although you
  never wrote the window anywhere.
\end{itemize}

\hypertarget{try-catch}{%
\paragraph{Try-Catch}\label{try-catch}}

\begin{itemize}
\tightlist
\item
  Try catch are two statements in Javascript that go together.
\item
  They have to do with errors and exceptions in Javascript.
\item
  Specifically, they have to do with catching errors and preventing them
  from breaking or stopping the execution of our code.
\item
  We can't have just try we also need to have catch which is a block of
  code that will run if there was an exception or error generated inside
  of the try block.
\item
  Generally, code after an error is not executed and the execution of
  the code stops at the point, the error is found.
\item
  However, using try/catch block we can avoid that.
\item
  Syntax:
  \texttt{JS\ \ \ \ \ try\{\ \ \ \ \ \ \ \ \ code;\ \ \ \ \ \}\ \ \ \ \ except\ ({[}e{]})\{\ \ \ \ \ \ \ \ \ code\ in\ case\ of\ error;\ \ \ \ \ \}}
\item
  Here e is optional, and e is the error which usually is printed out in
  case of an error. We can also print the error if required.
\end{itemize}

\hypertarget{arrays-and-callback-methods}{%
\subsubsection{Arrays and Callback
Methods}\label{arrays-and-callback-methods}}

\hypertarget{for-each}{%
\paragraph{For each}\label{for-each}}

\begin{itemize}
\tightlist
\item
  Reference -
  https://developer.mozilla.org/en-US/docs/Web/JavaScript/Reference/Global\_Objects/Array/forEach
\item
  What for\_each does is present in the name itself, that is the
  for\_each calls the function for each element in the array.
\item
  Accepts a callback function. Calls the function once per element in
  the array.
\item
  It allows us to run some function i.e run some code once per item in
  some array.
\item
  So whatever function we pass in, that function will be called once per
  item where each item will be passed in to the function automatically.
\item
  \textbf{The parameter in the function represents each element of the
  array}.
\item
  For example (refer the code)
\item
  In the code, the movie is the parameter and it represents each element
  in the array.
\item
  Difference between for\_each and map -
\item
  https://codeburst.io/javascript-map-vs-foreach-f38111822c0f
\item
  In map, and filter, we cannot console.log() but rather we return the
  array to a new variable whereas in forEach we can console.log() or
  either mutate the array.
\end{itemize}

\hypertarget{map-method}{%
\paragraph{Map Method}\label{map-method}}

\begin{itemize}
\tightlist
\item
  Reference -
  https://developer.mozilla.org/en-US/docs/Web/JavaScript/Reference/Global\_Objects/Map
\item
  Map method creates a new array with the results of calling a callback
  on every element in the array.
\item
  Example
  \texttt{JS\ \ \ \ \ const\ texts\ =\ {[}\textquotesingle{}rofl\textquotesingle{},\ \textquotesingle{}lol\textquotesingle{},\ \textquotesingle{}omg\textquotesingle{},\ \textquotesingle{}ttyl\textquotesingle{}{]};\ \ \ \ \ const\ caps\ =\ texts.map(function(t)\{\ //The\ function\ parameter\ t\ represents\ each\ element\ of\ texts\ \ \ \ \ \ \ \ \ return\ t.toUpperCase();\ \ \ \ \ \})\ \ \ \ \ texts;\ //{[}"rofl","lol","omg","ttyl"{]}\ \ \ \ \ caps;\ {[}"ROFL",\ "LOL",\ "OMG",\ "TTYL"{]}}
\end{itemize}

\hypertarget{intro-to-arrow-functions}{%
\paragraph{Intro to Arrow Functions}\label{intro-to-arrow-functions}}

\begin{itemize}
\item
  Newer syntax for defining functions.
\item
  Syntactically compact alternative to a regular function expression.
\item
  Example ```JS const square = (x) =\textgreater\{ return x\emph{x; \}
  This can also be written as(However this works for functions with just
  one parameter) const square = x =\textgreater\{ return x}x; \}

  const sum = (x,y) =\textgreater{} \{ return x + y; \} ```
\end{itemize}

\hypertarget{implicit-arrow-function}{%
\paragraph{Implicit Arrow Function}\label{implicit-arrow-function}}

\begin{verbatim}
Implicit Return - All these functions do the same thing.
\end{verbatim}

\begin{enumerate}
\def\labelenumi{\arabic{enumi}.}
\tightlist
\item
  const isEven = function(num)\{ //regular function expression return
  num\%2==0 \}
\item
  const isEven = (num) =\textgreater{} \{ //arrow function with parens
  around param return num \% 2 == 0 \}
\item
  const isEven = num =\textgreater{} \{ //no parens around params return
  num \% 2 == 0 \}
\item
  const isEven = num =\textgreater{} ( //implicit return num \% 2 == 0
  );
\item
  const isEven = num =\textgreater{} num\%2 == 0; //one-linear implicit
  return
\item
  Implicit returns just works when there's a single clear value to be
  evaluated.
\item
  It doesn't work in the cases like this
  \texttt{JS\ \ \ \ \ const\ rollDie\ =\ ()\ =\textgreater{}\ \{\ \ \ \ \ let\ greet\ =\ "Hello"\ \ \ \ \ return\ Math.floor(Math.random()*6+1)\ \ \ \ \ \}}
\end{enumerate}

\hypertarget{sorting-out-functions}{%
\paragraph{Sorting out Functions}\label{sorting-out-functions}}

\begin{enumerate}
\def\labelenumi{\arabic{enumi}.}
\tightlist
\item
  Functions //Normal Functions function funcName({[}parameters{]})\{
  //do something. \}
\item
  Function Expressions //Another way of writing Functions variable =
  function({[}parameters{]})\{ //do something \}
\item
  Higher Order Functions //Functions that work with other functions and
  can accept other functions as arguments and return a function function
  funcName(funcParameter)\{ funcParameter(); \}
\item
  Returning functions // Can return a function but need to capture the
  value of the outer function and then call the inner function function
  funcName({[}parameters{]})\{ return function()\{ //Do something \} \}
\item
  Methods //Can add functions as properties on objects. variable objName
  = \{ variable key : value, variable key : function({[}parameters{]})\{
  code; \}, funcName({[}parameters{]})\{ //code \} \} objName.variable
  //in case of a value variable objName.variable({[}parameters{]}) //in
  case of a function variable
\item
  Arrow functions //Syntactically compact alternative to regular
  function expressions. variable = ({[}parameters{]}) =\textgreater\{
  //code return expression \} //You can also remove the parentheses in
  arrow functions in case of a single parameter
\item
  Implicit return //Just works when single values are to be evaluated or
  returned variable = (parameters) =\textgreater{} ( code //No need to
  write return statement ) variable = parameter =\textgreater{}
  expression //One linear implicit return
\end{enumerate}

\begin{itemize}
\tightlist
\item
  Different methods
\end{itemize}

\begin{enumerate}
\def\labelenumi{\arabic{enumi}.}
\tightlist
\item
  forEach //Allows us to run some function once per item in an array.
  list.forEach(function({[}parameter{]})\{ //Here the parameter is used
  to denote the elements in an array //code \}) list.forEach(function)
\item
  map //Allows us to create a new array with the results of calling a
  call back on every element on the array. variable = list.map(function)
\end{enumerate}

\hypertarget{settimeout-setinterval}{%
\paragraph{setTimeout, setInterval}\label{settimeout-setinterval}}

\begin{itemize}
\tightlist
\item
  \textbf{Introduction}
\end{itemize}

\begin{enumerate}
\def\labelenumi{\arabic{enumi}.}
\tightlist
\item
  These are the two functions that expects you to pass a callback
  function in but they are not array methods.They have nothing to do
  with arrays.
\item
  They are used for scheduling execution.
\item
  In other programming languages, it is known as sleep, pause where we
  pause execution for some parameter time however in Javascript it is
  not similar to pause(3000) or sleep(3000)
\item
  Reference for clearTimeout and clearInterval -
  https://www.geeksforgeeks.org/javascript-cleartimeout-clearinterval-method/
\end{enumerate}

\begin{itemize}
\tightlist
\item
  \textbf{setTimeout, setInterval, clearInterval}
\end{itemize}

\begin{enumerate}
\def\labelenumi{\arabic{enumi}.}
\tightlist
\item
  setTimeout This function executes the function after set interval of
  time. setTimeout(TimerHandler,timeout?:number) Here, the TimerHandler
  is a function basically and timeout is how long something should take
  or how long delay should be. 2.setInterval This function repeats the
  function after set interval of time. setInterval(TImerHandler,
  timeout)
\item
  clearInterval Everytime we call setInterval it gives an id or value
  and we need to save the return value of setInterval, we can have whole
  bunch of different setIntervals and we can specify which one we want
  to stop by using this ID. id = setInterval(TimerHandler, timeout)
  clearInterval(id)
\end{enumerate}

\hypertarget{filter-method}{%
\paragraph{Filter Method}\label{filter-method}}

\begin{itemize}
\tightlist
\item
  Creates a new array with all elements that pass the test implemented
  by the provided function.
\item
  Example
  \texttt{JS\ \ \ \ \ const\ nums\ =\ {[}9,8,7,6,5,4,3,2,1{]}\ \ \ \ \ \ const\ odds\ =\ nums.filter(n\ =\textgreater{}\ \{\ \ \ \ \ \ \ \ \ return\ n\%2\ ==\ 1;\ //our\ callback\ returns\ true\ or\ false\ \ \ \ \ \ \ \ \ //if\ it\ returns\ true,\ n\ is\ added\ to\ filtered\ array\ \ \ \ \ \})\ \ \ \ \ //{[}9,7,5,3,1{]}}
\item
  Example 2
  \texttt{JS\ \ \ \ \ const\ smallNums\ =\ nums.filter(n\ =\textgreater{}\ n\textless{}5);\ \ \ \ \ {[}4,3,2,1{]}}
\item
  So in short if the function passed to the filter returns true for a
  given element, filter will return that particular element.
\end{itemize}

\hypertarget{some-and-every}{%
\paragraph{Some and Every}\label{some-and-every}}

\begin{itemize}
\tightlist
\item
  Both of them are very similar and they are boolean methods meaning
  they return true or false
\end{itemize}

\begin{enumerate}
\def\labelenumi{\arabic{enumi}.}
\tightlist
\item
  Sum method

  \begin{enumerate}
  \def\labelenumii{\arabic{enumii}.}
  \tightlist
  \item
    tests whether some elements in the array pass the provided function.
    It returns a Boolean value.
  \end{enumerate}
\item
  Every method

  \begin{enumerate}
  \def\labelenumii{\arabic{enumii}.}
  \tightlist
  \item
    tests whether all elements in the array pass the provided function.
    It returns a Boolean value.
  \end{enumerate}
\item
  Example ```JS const words = {[}``dog'', ``dig'', ``log'', ``bag'',
  ``wag''{]};

  \begin{enumerate}
  \def\labelenumii{\arabic{enumii}.}
  \item
    Every words.every(word =\textgreater{} \{ return word.length == 3;
    \})//true

    Some words.some(word =\textgreater{} \{ return word.length == 3;
    \})//true
  \item
    Every words.every(word =\textgreater{} word{[}0{]} === ``d'')
    //false

    Some words.some(word =\textgreater{} word{[}0{]} === ``d'') //true
  \item
    Every words.every(w =\textgreater\{ let last\_letter =
    w{[}w.length-1{]} return last\_letter === ``g''; \}) //true

    Some words.every(w =\textgreater\{ let last\_letter =
    w{[}w.length-1{]} return last\_letter === ``g''; \}) //true ```
  \end{enumerate}
\end{enumerate}

\hypertarget{reduce}{%
\paragraph{Reduce}\label{reduce}}

\begin{itemize}
\tightlist
\item
  Executes a reducer function on each element of the array, resulting in
  a single value.
\item
  Example
  \texttt{JS\ \ \ \ \ {[}3,5,7,9,11{]}.reduce((accumulator,\ currentvalue)\ =\textgreater{}\{\ \ \ \ \ \ \ \ \ return\ accumulator\ +\ currentvalue;\ \ \ \ \ \})}
\item
  Accumulator will be a thing, which we are reducing down to.
  Accumulator variable in above case will hold the sum, current value
  represents each individual element of the array.
\item
  Reduce in javascript works in the same way as that of Python, that is
  it does sequential computation.
\item
  Accumulator will be initially the first element of the array, which in
  the above case, is 3 and then gets changed according to the operation
  with the current value and currentvalue will get swapped after each
  callback and in the above case they are 5,7,9 and 11.
\end{itemize}

\hypertarget{this-in-arrow-functions-refer-code}{%
\paragraph{This in Arrow Functions (Refer
Code)}\label{this-in-arrow-functions-refer-code}}

\begin{itemize}
\item
  \textbf{this keyword in arrow functions}
\item
  `this' keyword in arrow functions behaves much differently in arrow
  functions than in normal functions.
\item
  Inside of an arrow function, the keyword this is just going to refer
  to the scope that it was created in.
\item
  So in this case, the keyword this refers to the window object just
  similar to the previous section where we pointed an object function to
  a new variable.
\item
  \textbf{Understanding the arrow\_this.js code}
\item
  person is the object
\item
  thisPrinter is a NORMAL function which returns person object on
  printing this.
\item
  fullName function is also a NORMAL function which also refers to the
  person object.
\item
  Now, fullName2() is an ARROW function, and it refers to the window
  object. Therefore, this.firstName and this.lastName will print as
  undefined.
\item
  In shoutName, the NORMAL function shoutName actually refers to the
  person object BUT the NORMAL setTimeout function inside the shoutname
  function refers to window object.
\item
  Whereas in shoutName2, the setTimeout ARROW function has the same
  value of this as that of shoutName2 function and hence it will refer
  to the person object.
\item
  Therefore in short,
\item
  A nested NORMAL function or a NORMAL function in a NORMAL function
  does not refer to the same `this' keyword that the parent function is
  referring and hence it refers to window object. //code line 37 and
  here for this.firstName it will print it's parent function firstName
  that is hello, in case of firstName not defined in parent function, it
  will print undefined.
\item
  A nested arrow function or an arrow function in a NORMAL function
  refers to the same `this' keyword that the parent function is
  referring. //code line 45
\item
  Arrow functions by themselves in an object does not refer `this' to
  the object and hence it refers to the window object.
\item
  Therefore normal functions must be used in an object in order to use
  the this keyword however if you want to use `this' in a nested
  function, you may use arrow functions.
\end{itemize}

\hypertarget{arrays-and-callback-methods-1}{%
\subsubsection{Arrays and Callback
Methods}\label{arrays-and-callback-methods-1}}

\hypertarget{for-each-1}{%
\paragraph{For each}\label{for-each-1}}

\begin{itemize}
\tightlist
\item
  Reference -
  https://developer.mozilla.org/en-US/docs/Web/JavaScript/Reference/Global\_Objects/Array/forEach
\item
  What for\_each does is present in the name itself, that is the
  for\_each calls the function for each element in the array.
\item
  Accepts a callback function. Calls the function once per element in
  the array.
\item
  It allows us to run some function i.e run some code once per item in
  some array.
\item
  So whatever function we pass in, that function will be called once per
  item where each item will be passed in to the function automatically.
\item
  \textbf{The parameter in the function represents each element of the
  array}.
\item
  For example (refer the code)
\item
  In the code, the movie is the parameter and it represents each element
  in the array.
\item
  \href{https://codeburst.io/javascript-map-vs-foreach-f38111822c0f}{Difference
  between for\_each and map}
\item
  In map, and filter, we cannot console.log() but rather we return the
  array to a new variable whereas in forEach we can console.log() or
  either mutate the array.
\end{itemize}

\end{document}
